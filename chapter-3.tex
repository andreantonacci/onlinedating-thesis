
\lhead[\leftmark]{\leftmark}

\rhead[\leftmark]{}

\lfoot{}

\rfoot{}

\cfoot[\thepage]{\thepage}

\chapter{Results}

\section{Descriptive Statistics}

We observe that females in our sample are 51.5\% and that the average
age is 46.7. Both \textsc{age} and \textsc{polview} seem normally
distributed \textendash{} the absolute value of skewness and kurtosis
are < 1 and few outliers have been observed \textendash{} while \textsc{internetfreq}
seems left-skewed (see Figures \ref{fig:age:-histogram.}, \ref{fig:polview-histogram},
\ref{fig:internetfreq-histogram} and \prettyref{tab:Descriptive-Statistics-forNumerical}):
it has a mean of 3.76 but skewness is slightly problematic (-1.087).
We encountered the same kind of problem with other variables. However,
due to the very large sample size, we considered that the Central
Limit Theorem could be applied and therefore we refused to drop these
important variables. 

No significant issues emerged from a Missing Value Analysis (see \prettyref{tab:Missing-Value-Analysis-Independ}),
with few notable exceptions\textsc{: internetfreq (}14.1\% of missing
values\textsc{), physicallabour (}41.4\%\textsc{), polview (}9.6\%\textsc{)}
and\textsc{ married (}19.2\%\textsc{). }
\begin{table}[h]
\centering{}%
\begin{minipage}[t]{1\columnwidth}%
\begin{center}
\renewcommand{\arraystretch}{1.5}%
\begin{tabular}{lccccccc}
\hline 
 &  &  & \textbf{Std. } & \multicolumn{2}{c}{\textbf{Missing}} & \multicolumn{2}{c}{\textbf{N. of Ext.}\footnote{Number of extreme cases outside the range (Q1 - 1.5{*}IQR, Q3 + 1.5{*}IQR). }\textbf{}\footnote{. indicates that the inter-quartile range (IQR) is zero.}}\tabularnewline
 & \textbf{N} & \textbf{Mean} & \textbf{Deviation} & \textbf{Count} & \textbf{Percent} & \textbf{Low} & \textbf{High}\tabularnewline
\hline 
\hline 
\textsc{age} & 5318 & 46.2232 & 17.84838 & 100 & 1.8 & 0 & 0\tabularnewline
\textsc{gender} & 5418 & .5172 & .49975 & 0 & .0 & 0 & 0\tabularnewline
\textsc{university} & 5383 & .2631 & .44033 & 35 & .6 & 0 & 0\tabularnewline
\textsc{minrace} & 5291 & .2657 & .44177 & 127 & 2.3 & . & .\tabularnewline
\textsc{income1} & 5418 & .3073 & .46142 & 0 & .0 & 0 & 0\tabularnewline
\textsc{income2} & 5418 & .2999 & .45827 & 0 & .0 & 0 & 0\tabularnewline
\textsc{income3} & 5418 & .2527 & .43459 & 0 & .0 & 0 & 0\tabularnewline
\textsc{married} & 4378 & .6942 & .46082 & 1040 & 19.2 & 0 & 0\tabularnewline
\textsc{isparent} & 5365 & .2939 & .45561 & 53 & 1.0 & . & .\tabularnewline
\textsc{polview} & 4897 & 2.9579 & 1.05925 & 521 & 9.6 & 0 & 0\tabularnewline
\textsc{urban} & 5418 & .8319 & .37403 & 0 & .0 & . & .\tabularnewline
\textsc{physicallabour} & 3173 & .4901 & .49998 & 2245 & 41.4 & 0 & 0\tabularnewline
\textsc{disability} & 5399 & .1721 & .37748 & 19 & .4 & . & .\tabularnewline
\textsc{internetfreq} & 4652 & 3.7623 & 1.13325 & 766 & 14.1 & 136 & 0\tabularnewline
\textsc{attitude1} & 5094 & .6307 & .48265 & 324 & 6.0 & 0 & 0\tabularnewline
\textsc{attitude2} & 4985 & .5047 & .50003 & 433 & 8.0 & 0 & 0\tabularnewline
\hline 
\end{tabular}\medskip{}
\par\end{center}
\begin{center}
\caption{Missing Value Analysis.\label{tab:Missing-Value-Analysis-Independ}}
\par\end{center}%
\end{minipage}
\end{table}
However, these high percentages could be explained by the fact that
these questions have not been asked to all respondents. In fact, \textsc{internetfreq}
has been asked only if the respondent had previously admitted to use
the Internet or own a cell phone; similarly, \textsc{physicallabour}
has been asked only to currently employed people (leaving out not
only the elderly, but also students or recent graduates). On the other
hand, questions on sensitive matters, like political orientation,
usually entail a higher than average percentage of missing values.
\textsc{married}, instead, has been recoded to take value 1 (\textquotedblleft Yes\textquotedblright )
only if the respondent is currently married or living with a partner,
and 0 (\textquotedblleft No\textquotedblright ) if he or she has never
been married before, thus leaving out all divorced, separated and
widowed. Although the adequate percentage of missing values ranges
from 5\% to 8\% at most, we decided to keep these variables due to
their considerable high number of observations and their crucial importance
in our model.

\prettyref{tab:Correlation-matrix} displays the bivariate (linear)
correlations between all pairs of variables selected in our model.
This reveals some interesting and statistically significant links
among the variables, that to a certain extent seem to confirm our
\textsl{hypothesis 2}. For instance, we notice that online dating
is positively correlated with being male, young, highly educated,
with liberal views, single and with no children. Interestingly enough,
some insights on U.S. (and sometimes worldwide) common social issues
emerge as well. \clearpage{}\pagestyle{plain}
\begin{sidewaystable}
\centering{}%
\begin{minipage}[t]{1\columnwidth}%
\begin{center}
\renewcommand{\arraystretch}{0.8}%
\begin{tabular}{|l|l|c|c|c|c|c|c|c|c|c|c|c|c|c|c|c|c|c|}
\hline 
\multicolumn{1}{l}{} & \multicolumn{1}{l}{} & \multicolumn{1}{c}{\textsc{\scriptsize{}ison.dat.}} & \multicolumn{1}{c}{\textsc{\scriptsize{}gen.}} & \multicolumn{1}{c}{\textsc{\scriptsize{}age}} & \multicolumn{1}{c}{\textsc{\scriptsize{}uni}} & \multicolumn{1}{c}{\textsc{\scriptsize{}m.race}} & \multicolumn{1}{c}{\textsc{\scriptsize{}inc.1}} & \multicolumn{1}{c}{\textsc{\scriptsize{}inc.2}} & \multicolumn{1}{c}{\textsc{\scriptsize{}inc.3}} & \multicolumn{1}{c}{\textsc{\scriptsize{}marr.}} & \multicolumn{1}{c}{\textsc{\scriptsize{}ispar.}} & \multicolumn{1}{c}{\textsc{\scriptsize{}polview}} & \multicolumn{1}{c}{\textsc{\scriptsize{}urb.}} & \multicolumn{1}{c}{\textsc{\scriptsize{}phy.lab.}} & \multicolumn{1}{c}{\textsc{\scriptsize{}disa.}} & \multicolumn{1}{c}{\textsc{\scriptsize{}int.freq}} & \multicolumn{1}{c}{\textsc{\scriptsize{}att.1}} & \multicolumn{1}{c}{\textsc{\scriptsize{}att.2}}\tabularnewline
\hline 
\hline 
\multirow{2}{*}{\textsc{\scriptsize{}ison.dat.}} & {\tiny{}Pearson C.} & {\tiny{}1} &  &  &  &  &  &  &  &  &  &  &  &  &  &  &  & \tabularnewline
\cline{2-19} 
 & {\tiny{}Sig.} &  &  &  &  &  &  &  &  &  &  &  &  &  &  &  &  & \tabularnewline
\hline 
\multirow{2}{*}{\textsc{\scriptsize{}gen.}} & {\tiny{}Pearson C.} & {\tiny{}-.048{*}{*}} & {\tiny{}1} &  &  &  &  &  &  &  &  &  &  &  &  &  &  & \tabularnewline
\cline{2-19} 
 & {\tiny{}Sig.} & {\tiny{}.000} &  &  &  &  &  &  &  &  &  &  &  &  &  &  &  & \tabularnewline
\hline 
\multirow{2}{*}{\textsc{\scriptsize{}age}} & {\tiny{}Pearson C.} & {\tiny{}-.211{*}{*}} & {\tiny{}.045{*}{*}} & {\tiny{}1} &  &  &  &  &  &  &  &  &  &  &  &  &  & \tabularnewline
\cline{2-19} 
 & {\tiny{}Sig.} & {\tiny{}.000} & {\tiny{}.000} &  &  &  &  &  &  &  &  &  &  &  &  &  &  & \tabularnewline
\hline 
\multirow{2}{*}{\textsc{\scriptsize{}uni}} & {\tiny{}Pearson C.} & {\tiny{}.056{*}{*}} & {\tiny{}.013} & {\tiny{}.044{*}{*}} & {\tiny{}1} &  &  &  &  &  &  &  &  &  &  &  &  & \tabularnewline
\cline{2-19} 
 & {\tiny{}Sig.} & {\tiny{}.000} & {\tiny{}.311} & {\tiny{}.001} &  &  &  &  &  &  &  &  &  &  &  &  &  & \tabularnewline
\hline 
\multirow{2}{*}{\textsc{\scriptsize{}m.race}} & {\tiny{}Pearson C.} & {\tiny{}-.018} & {\tiny{}-.011} & {\tiny{}-.143{*}{*}} & {\tiny{}-.048{*}{*}} & {\tiny{}1} &  &  &  &  &  &  &  &  &  &  &  & \tabularnewline
\cline{2-19} 
 & {\tiny{}Sig.} & {\tiny{}.167} & {\tiny{}.372} & {\tiny{}.000} & {\tiny{}.000} &  &  &  &  &  &  &  &  &  &  &  &  & \tabularnewline
\hline 
\multirow{2}{*}{\textsc{\scriptsize{}inc.1}} & {\tiny{}Pearson C.} & {\tiny{}.000} & {\tiny{}.046{*}{*}} & {\tiny{}-.002} & {\tiny{}-.253{*}{*}} & {\tiny{}.129{*}{*}} & {\tiny{}1} &  &  &  &  &  &  &  &  &  &  & \tabularnewline
\cline{2-19} 
 & {\tiny{}Sig.} & {\tiny{}.991} & {\tiny{}.000} & {\tiny{}.880} & {\tiny{}.000} & {\tiny{}.000} &  &  &  &  &  &  &  &  &  &  &  & \tabularnewline
\hline 
\multirow{2}{*}{\textsc{\scriptsize{}inc.2}} & {\tiny{}Pearson C.} & {\tiny{}.045{*}{*}} & {\tiny{}-.019} & {\tiny{}-.032{*}} & {\tiny{}-.005} & {\tiny{}-.023} & {\tiny{}-.429} & {\tiny{}1} &  &  &  &  &  &  &  &  &  & \tabularnewline
\cline{2-19} 
 & {\tiny{}Sig.} & {\tiny{}.000} & {\tiny{}.137} & {\tiny{}.012} & {\tiny{}.704} & {\tiny{}.074} & {\tiny{}.000} &  &  &  &  &  &  &  &  &  &  & \tabularnewline
\hline 
\multirow{2}{*}{\textsc{\scriptsize{}inc.3}} & {\tiny{}Pearson C.} & {\tiny{}.023} & {\tiny{}-.057{*}{*}} & {\tiny{}-.052{*}{*}} & {\tiny{}.301{*}{*}} & {\tiny{}-.075{*}{*}} & {\tiny{}-.388{*}{*}} & {\tiny{}-.386{*}{*}} & {\tiny{}1} &  &  &  &  &  &  &  &  & \tabularnewline
\cline{2-19} 
 & {\tiny{}Sig.} & {\tiny{}.067} & {\tiny{}.000} & {\tiny{}.000} & {\tiny{}.000} & {\tiny{}.000} & {\tiny{}.000} & {\tiny{}.000} &  &  &  &  &  &  &  &  &  & \tabularnewline
\hline 
\multirow{2}{*}{\textsc{\scriptsize{}marr.}} & {\tiny{}Pearson C.} & {\tiny{}-.282{*}{*}} & {\tiny{}.038{*}{*}} & {\tiny{}.478{*}{*}} & {\tiny{}.126{*}{*}} & {\tiny{}-.145{*}{*}} & {\tiny{}-.177{*}{*}} & {\tiny{}.038{*}{*}} & {\tiny{}-.386{*}{*}} & {\tiny{}1} &  &  &  &  &  &  &  & \tabularnewline
\cline{2-19} 
 & {\tiny{}Sig.} & {\tiny{}.000} & {\tiny{}.000} & {\tiny{}.000} & {\tiny{}.000} & {\tiny{}.000} & {\tiny{}.000} & {\tiny{}.008} & {\tiny{}.000} &  &  &  &  &  &  &  &  & \tabularnewline
\hline 
\multirow{2}{*}{\textsc{\scriptsize{}ispar.}} & {\tiny{}Pearson C.} & {\tiny{}-.047{*}{*}} & {\tiny{}.046{*}{*}} & {\tiny{}-.263{*}{*}} & {\tiny{}.020} & {\tiny{}.056{*}{*}} & {\tiny{}-.006} & {\tiny{}.000} & {\tiny{}.090{*}{*}} & {\tiny{}.249{*}{*}} & {\tiny{}1} &  &  &  &  &  &  & \tabularnewline
\cline{2-19} 
 & {\tiny{}Sig.} & {\tiny{}.000} & {\tiny{}.000} & {\tiny{}.000} & {\tiny{}.119} & {\tiny{}.000} & {\tiny{}.637} & {\tiny{}.983} & {\tiny{}.000} & {\tiny{}.000} &  &  &  &  &  &  &  & \tabularnewline
\hline 
\multirow{2}{*}{\textsc{\scriptsize{}polview}} & {\tiny{}Pearson C.} & {\tiny{}.103{*}{*}} & {\tiny{}.049{*}{*}} & {\tiny{}-.145{*}{*}} & {\tiny{}.096{*}{*}} & {\tiny{}.084{*}{*}} & {\tiny{}.033{*}} & {\tiny{}-.009} & {\tiny{}.025} & {\tiny{}-.126{*}{*}} & {\tiny{}-.010} & {\tiny{}1} &  &  &  &  &  & \tabularnewline
\cline{2-19} 
 & {\tiny{}Sig.} & {\tiny{}.000} & {\tiny{}.000} & {\tiny{}.000} & {\tiny{}.000} & {\tiny{}.000} & {\tiny{}.014} & {\tiny{}.499} & {\tiny{}.058} & {\tiny{}.000} & {\tiny{}.435} &  &  &  &  &  &  & \tabularnewline
\hline 
\multirow{2}{*}{\textsc{\scriptsize{}urb.}} & {\tiny{}Pearson C.} & {\tiny{}.047{*}{*}} & {\tiny{}.018} & {\tiny{}-.117{*}{*}} & {\tiny{}.101{*}{*}} & {\tiny{}.109{*}{*}} & {\tiny{}-.045{*}{*}} & {\tiny{}-.041{*}{*}} & {\tiny{}-.078{*}{*}} & {\tiny{}-.109{*}{*}} & {\tiny{}-.049{*}{*}} & {\tiny{}.091{*}{*}} & {\tiny{}1} &  &  &  &  & \tabularnewline
\cline{2-19} 
 & {\tiny{}Sig.} & {\tiny{}.000} & {\tiny{}.147} & {\tiny{}.000} & {\tiny{}.000} & {\tiny{}.000} & {\tiny{}.000} & {\tiny{}.001} & {\tiny{}.000} & {\tiny{}.000} & {\tiny{}.000} & {\tiny{}.000} &  &  &  &  &  & \tabularnewline
\hline 
\multirow{2}{*}{\textsc{\scriptsize{}phy.lab.}} & {\tiny{}Pearson C.} & {\tiny{}-.041{*}} & {\tiny{}-.107{*}{*}} & {\tiny{}-.123{*}{*}} & {\tiny{}-.453{*}{*}} & {\tiny{}.098{*}{*}} & {\tiny{}.222{*}{*}} & {\tiny{}.070{*}{*}} & {\tiny{}-.260{*}{*}} & {\tiny{}-.134{*}{*}} & {\tiny{}.021} & {\tiny{}-.004} & {\tiny{}-.080{*}{*}} & {\tiny{}1} &  &  &  & \tabularnewline
\cline{2-19} 
 & {\tiny{}Sig.} & {\tiny{}.012} & {\tiny{}.000} & {\tiny{}.000} & {\tiny{}.000} & {\tiny{}.000} & {\tiny{}.000} & {\tiny{}.000} & {\tiny{}.000} & {\tiny{}.000} & {\tiny{}.208} & {\tiny{}.830} & {\tiny{}.000} &  &  &  &  & \tabularnewline
\hline 
\multirow{2}{*}{\textsc{\scriptsize{}disa.}} & {\tiny{}Pearson C.} & {\tiny{}-.015} & {\tiny{}.046{*}{*}} & {\tiny{}.226{*}{*}} & {\tiny{}-.157{*}{*}} & {\tiny{}.011} & {\tiny{}.250{*}{*}} & {\tiny{}-.077{*}{*}} & {\tiny{}-.184{*}{*}} & {\tiny{}.017} & {\tiny{}-.120{*}{*}} & {\tiny{}-.022} & {\tiny{}-.043{*}{*}} & {\tiny{}.016} & {\tiny{}1} &  &  & \tabularnewline
\cline{2-19} 
 & {\tiny{}Sig.} & {\tiny{}.231} & {\tiny{}.000} & {\tiny{}.000} & {\tiny{}.000} & {\tiny{}.390} & {\tiny{}.000} & {\tiny{}.000} & {\tiny{}.000} & {\tiny{}.228} & {\tiny{}.000} & {\tiny{}.100} & {\tiny{}.001} & {\tiny{}.334} &  &  &  & \tabularnewline
\hline 
\multirow{2}{*}{\textsc{\scriptsize{}int.freq}} & {\tiny{}Pearson C.} & {\tiny{}.152{*}{*}} & {\tiny{}-.012} & {\tiny{}-.344{*}{*}} & {\tiny{}.162{*}{*}} & {\tiny{}.013} & {\tiny{}-.195{*}{*}} & {\tiny{}.017} & {\tiny{}.178{*}{*}} & {\tiny{}-.077{*}{*}} & {\tiny{}.101{*}{*}} & {\tiny{}.111{*}{*}} & {\tiny{}.105{*}{*}} & {\tiny{}-.243{*}{*}} & {\tiny{}-.174{*}{*}} & {\tiny{}1} &  & \tabularnewline
\cline{2-19} 
 & {\tiny{}Sig.} & {\tiny{}.000} & {\tiny{}.360} & {\tiny{}.000} & {\tiny{}.000} & {\tiny{}.351} & {\tiny{}.000} & {\tiny{}.220} & {\tiny{}.000} & {\tiny{}.000} & {\tiny{}.000} & {\tiny{}.000} & {\tiny{}.000} & {\tiny{}.000} & {\tiny{}.000} &  &  & \tabularnewline
\hline 
\multirow{2}{*}{\textsc{\scriptsize{}att.1}} & {\tiny{}Pearson C.} & {\tiny{}.167{*}{*}} & {\tiny{}-.096{*}{*}} & {\tiny{}-.161{*}{*}} & {\tiny{}.074{*}{*}} & {\tiny{}-.079{*}{*}} & {\tiny{}-.091{*}{*}} & {\tiny{}.031{*}} & {\tiny{}.112{*}{*}} & {\tiny{}-.097{*}{*}} & {\tiny{}-.015} & {\tiny{}.054{*}{*}} & {\tiny{}.021} & {\tiny{}-.084{*}{*}} & {\tiny{}-.051{*}{*}} & {\tiny{}.156{*}{*}} & {\tiny{}1} & \tabularnewline
\cline{2-19} 
 & {\tiny{}Sig.} & {\tiny{}.000} & {\tiny{}.000} & {\tiny{}.000} & {\tiny{}.000} & {\tiny{}.000} & {\tiny{}.000} & {\tiny{}.019} & {\tiny{}.000} & {\tiny{}.000} & {\tiny{}.243} & {\tiny{}.000} & {\tiny{}.109} & {\tiny{}.000} & {\tiny{}.000} & {\tiny{}.000} &  & \tabularnewline
\hline 
\multirow{2}{*}{\textsc{\scriptsize{}att.2}} & {\tiny{}Pearson C.} & {\tiny{}.113{*}{*}} & {\tiny{}-.062{*}{*}} & {\tiny{}-.062{*}{*}} & {\tiny{}.062{*}{*}} & {\tiny{}-.071{*}{*}} & {\tiny{}-.036{*}{*}} & {\tiny{}-.008} & {\tiny{}.067{*}{*}} & {\tiny{}-.030{*}} & {\tiny{}-.021} & {\tiny{}.018} & {\tiny{}.019} & {\tiny{}-.131{*}{*}} & {\tiny{}.024} & {\tiny{}.110{*}{*}} & {\tiny{}.400{*}{*}} & {\tiny{}1}\tabularnewline
\cline{2-19} 
 & {\tiny{}Sig.} & {\tiny{}.000} & {\tiny{}.000} & {\tiny{}.000} & {\tiny{}.000} & {\tiny{}.000} & {\tiny{}.007} & {\tiny{}.550} & {\tiny{}.000} & {\tiny{}.041} & {\tiny{}.113} & {\tiny{}.201} & {\tiny{}.145} & {\tiny{}.000} & {\tiny{}.067} & {\tiny{}.000} & {\tiny{}.000} & \tabularnewline
\hline 
\end{tabular}
\par\end{center}
\begin{center}
\caption{Bivariate correlation matrix.\label{tab:Correlation-matrix}}
\blfootnote{{**Correlation is significant at 0.01 level (2-tailed).}}\blfootnote{{*Correlation is significant at 0.05 level (2-tailed).}}
\par\end{center}%
\end{minipage}
\end{sidewaystable}
\clearpage{}\pagestyle{fancy}

Namely, the gender pay gap seems noticeable, by observing the correlations
between \textsc{gender} and different income levels. There is also
a significant linear relationship between being in a racial minority
and earning less and marrying less often. Finally, we note not only
that \textsc{internetfreq} is positively correlated with \textsc{isonlinedater},
but also that it averagely follows its same pattern of correlations
with other variables (i.e., it has a positive relation with being
male, young, richer, etc.). By contrast, the thin market \textsl{hypothesis
1} finds very little support \textendash{} if any \textendash{} from
this bivariate analysis since, for instance, living in a urban area
is positively correlated with \textsc{isonlinedater}, and having a
disability is negatively correlated with the frequency of Internet
usage. However, more meaningful and precise results on actual dependency
links are addressed with the following multiple regression models.


\section{Multivariate Analysis}

\prettyref{tab:MRM-benchmark} displays all the resulting models from
the multiple regressions that we have conducted. F test is significant
at 1\% in all models (see \prettyref{tab:ANOVA-table}). Unless otherwise
noted, all variables are statistically significant at the 0.01 level
as well, and no particular issues of multicollinearity have been detected
\textendash{} VIF scores never higher than $\simeq1.8$.

The first basic model only included six controlling variables. Each
coefficient brought significant results ($P-value<0.01$), except
for \textsc{income2} ($P-value<0.05$) and \textsc{income3} (which
becomes however significant in other models). The overall model accounted
for approximately 5.5\% of the variance of online dating usage ($R^{2}=.055$).
Moreover, all the variables\textquoteright{} coefficients seem consistent
with the theoretical framework built for \textsl{hypothesis 2} (except
for \textsc{income3}): thus it seems that being male, younger, with
a university degree, White and in the upper social classes, predicts
higher chances of being an online dater. In order to boost the explanatory
power of the model, we included some more variables.

Model 2, taking into consideration some attributes of the life course,
provides a considerable higher explanatory power ($Adjusted\;R^{2}=0.111$).
As expected, we find that one standard deviation decrease in \textsc{married}
has the strongest effect (23.7\% in Model 2 and 25.9\% in Model 5)
on our dependent variable compared to one standard deviation variation
in any other covariate. A third and a fourth more robust models are
subsequently developed, adding regressors on political views, disability
and then Internet attitudes. Model 4 reaches an $R^{2}$ of .123 thanks
to the contribution of the Internet factors, that are all statistically
significant at 1\% and with a positive sign. The latter model seems
to us the most useful and easily interpretable one. In fact, in Model
5, \textsc{physicallabour} has also been added, and although it further
increased $R^{2}$ (.126), we find more covariates to be statically
not significant at all (e.g., \textsc{age} or \textsc{isparent}) and
generally more counterintuitive results.

In pursuit of more powerful predictions, we also built a sixth model
(not shown in Table \ref{tab:MRM-benchmark}) with some interaction
terms: \textsc{genderxuniversity} and \textsc{genderxminrace}. However,
despite the higher $R^{2}$ (.137), it failed at providing just as
many significant beta coefficients, thus we dismissed it. To clarify,
we value more those lower R-square models with higher P-values rather
than vice versa. If the only goal of the model was to make an accurate
prediction, this would be a flawed idea. Yet this is not the case,
since we are more interested in assessing whether a (perhaps small)
reliable relationship truly exists among the considered variables.
Even though the model does not explain much of the variation of the
data, it still is significant. Moreover, it is quite common to find
this kind of R-square percentages in social science research. Outcomes
could be dictated by many other latent factors at play that may affect
other facets of the model \textendash{} especially if the data deal
with human behaviour, attitudes, thoughts, preferences and feelings,
which are often object of measurement errors.

One final remark on the model H\textsubscript{1} that has been designed
to test the thin market hypothesis. It is overall significant and
no issues have been detected as in the previous models. The three
key variables chosen to test the hypothesis \textendash{} \textsc{minrace},
\textsc{disability} and \textsc{urban} \textendash{} are all significant
at 0.01 level. Some valuable insights emerged and they are extensively
addressed in the following section.

\begin{table}[h]
\centering{}%
\begin{minipage}[t]{1\columnwidth}%
\begin{center}
\renewcommand{\arraystretch}{2}%
\begin{tabular}{lcccccc}
\hline 
 & {\small{}Model }\textbf{\small{}1} & {\small{}Model }\textbf{\small{}2} & {\small{}Model }\textbf{\small{}3} & {\small{}Model }\textbf{\small{}4} & {\small{}Model }\textbf{\small{}5} & {\small{}Model}\textbf{\small{} H}\textsubscript{\textbf{\small{}1}}\tabularnewline
\hline 
\hline 
\textsc{gender (}\textsl{female}\textsc{)} & -.038{*}{*} & -.048{*}{*} & -.044{*}{*} & -.047{*}{*} & -.063{*}{*} & -.041{*}{*}\tabularnewline
\textsc{age} & -.221{*}{*} & -.131{*}{*} & -.121{*}{*} & -.032 & .001 & -.229{*}{*}\tabularnewline
\textsc{university} & .064{*}{*} & .078{*}{*} & .063{*}{*} & .046{*}{*} & .047{*} & .068{*}{*}\tabularnewline
\textsc{minrace} & -.045{*}{*} & -.051{*}{*} & -.051{*}{*} & -.048{*}{*} & -.060{*}{*} & -.050{*}{*}\tabularnewline
\textsc{income2 v. income1} & .031{*} & .076{*}{*} & .082{*}{*} & .062{*}{*} & .060{*}{*} & \tabularnewline
\textsc{income3 v. income1} & -.006 & .052{*}{*} & .056{*}{*} & .035 & .060{*} & \tabularnewline
\textsc{married} &  & -.237{*}{*} & -.239{*}{*} & -.259{*}{*} & -.243{*}{*} & \tabularnewline
\textsc{isparent} &  & -.052{*}{*} & -.045{*}{*} & -.025 & -.069{*}{*} & \tabularnewline
\textsc{polview (}\textsl{liberal}\textsc{)} &  &  & .076{*}{*} & .066{*}{*} & .122{*}{*} & \tabularnewline
\textsc{disability} &  &  & .028 & .037{*} & -.011 & .049{*}{*}\tabularnewline
\textsc{internetfreq} &  &  &  & .062{*}{*} & .052{*} & \tabularnewline
\textsc{attitude1} &  &  &  & .084{*}{*} & .104{*}{*} & \tabularnewline
\textsc{attitude2} &  &  &  & .055{*}{*} & .037 & \tabularnewline
\textsc{physicallabour} &  &  &  &  & -.009 & \tabularnewline
\textsc{urban} &  &  &  &  &  & .030{*}\tabularnewline
 &  &  &  &  &  & \tabularnewline
\hline 
\textbf{No. of observations} & 6010 & 4849 & 4468 & 3678 & 2556 & 6000\tabularnewline
\textbf{R}\textsuperscript{\textbf{2}} & .055 & .112 & .119 & .123 & .126 & .056\tabularnewline
\textbf{Adjusted R}\textsuperscript{\textbf{2}} & .054 & .111 & .117 & .120 & .121 & .055\tabularnewline
\hline 
\end{tabular}\bigskip{}
\par\end{center}
\begin{center}
\caption{Multiple regression models on the likelihood of online dating.\label{tab:MRM-benchmark}}
\blfootnote{{DV:  \textsc{isonlinedater}. We are reporting the standardised beta coefficients only.}}\blfootnote{{**Significant at 0.01 level.}}\blfootnote{{*Significant at 0.05 level.}}
\par\end{center}%
\end{minipage}
\end{table}


