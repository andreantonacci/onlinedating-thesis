
\lhead[\leftmark]{\leftmark}

\rhead[\leftmark]{}

\lfoot{}

\rfoot{}

\cfoot[\thepage]{\thepage}

\chapter{Discussion}

\section{Common User Profiles}

Despite its poor robustness, model H\textsubscript{1} reveals some
interesting and rather contrasting insights. We partially reject \textsl{hypothesis
1} since the model predicted higher chances of being online daters
if White and living in a urban area \textendash{} i.e., not being
in a racial minority and living in a rural community far from the
large pool of potential partners of a big city. The positive sign
of \textsc{urban}, by violating the first hypothesis, actually validates
the concept of network effects, since people seem to use online dating
more in highly dense environments. Moreover, \textsc{age} in this
model has the strongest effect among all covariates (-.229\textsuperscript{{*}{*}})
and this goes further against the first hypothesis. Older singles
are, in fact, in a thin dating market, since most of their own kind
are already partnered in their 30s and 40s usually \citep{Michael-J.-Rosenfeld2012Searching-for-a}.
By contrast, \textsc{disability} appears to confirm the thin market
hypothesis with a standardised beta coefficient of .049, significant
at 1\%. This suggests that people with a handicap are in fact more
likely to resort to online dating than those who do not. Assessing
whether one is in a thin market for potential partners obviously requires
far more variables to be taken into account, in order to uncover the
latent factors behind such diverse human behaviours, attitudes and
preferences. All things considered, we do not have enough (and appropriate)
data to produce unambiguous findings on this topic, thus we dismiss
the first hypothesis. 

As regards \textsl{hypothesis 2}, results from the models seem consistent
with the theoretical framework previously discussed and in accordance
with the extensive studies already conducted on this topic. As we
would expect, most online daters are generally the youngest White
males with a higher income and education level than average. They
are, in fact, those who also spend more time on the Internet and have
more progressive views towards it. These people also tend to lean
more on democratic and liberal views in politics. According to our
models, online daters are generally more likely to have no children
nor a stable partner yet. Since only 5.8\% of married\footnote{I.e., 5.8\% of \textsc{married}. It includes all married people and
those living with a partner. } respondents also reported having utilised online dating platforms,
we infer that the vast majority of them found a stable partner through
conventional mating markets (and perhaps those in the minority are
just some residual cases of self-reported infidelity). It seems plausible
to assume that most of them married before the diffusion of online
dating in the United States, thus it would be erroneous to interpret
these data as a defeat of such platforms. Given the above, we find
enough empirical evidence to support and confirm the second hypothesis,
which however begs for more in-depth investigations.


\section{Implications for Business and Further Research}

Studies like this are far from being purely theoretical. Online dating
analyses can help to grasp insights on the economic mechanisms behind
match formation and marriages. The consequences of these results for
businesses are twofold. First and foremost, it is of absolute importance
that companies interpret current socio-demographic trends, in order
to understand men, women and families of the future. The way people
interact, get to know potential partners, fall in love, forge profound
relationships and eventually create new families, has a dramatic impact
on everyday trades and society as a whole. In other words, understanding
the logics behind mate choice processes also helps to understand the
logics behind consumer choices. Market metaphors are in this sense
useful to move companies closer to people and to what would be otherwise
considered naturally extraneous to exchange logics. A practical example
of taking advantage of these studies could be the more precise and
efficient use of programmatic advertising or, more generally, of targeted
ads on online dating platforms. 

One second key managerial concept mainly regards online dating companies.
Those market characterised with increasing returns of adoption, such
as network effects, steadily tend towards the dominance of a single
technology or product, usually in ways very difficult to reverse because
of lock-in effects. It is rare that more than a single product profitably
co-exist in the long term\footnote{E.g., think of the dominance of Facebook among social media.}.
Since very little differentiation can currently be found among different
online dating services in this phase, it is likely that in the future
one single dominant platform (or a dominant design) will emerge. Hence,
it seems crucial at this point to adopt measures safeguarding the
standalone value of one's own dating platform \textendash{} or to
attract as many users as possible. 

This work certainly raises more questions than it answers. Our models
work well for conveying simple concepts, yet they fail at exploring
all facets of intricate human behaviours and providing strong predictions.
Undoubtedly, actual user behaviours in both online and offline dating
markets cannot be exactly described nor predicted by models. Nevertheless,
we suggest further research to be carried out based on primary data
and taking into account more individual traits and distinctions, in
order to deeply explore the development of modern romantic relationships.
Additional analyses could not only inspect online dating market dynamics,
mate choice logics and marriages in complex networks, but they could
also qualitatively unravel the human stories behind those matches. 


