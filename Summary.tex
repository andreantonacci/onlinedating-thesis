
\chapter*{Abstract}

\addcontentsline{toc}{chapter}{Abstract} 

This thesis explores the efficiency gains in partner markets deriving
from the diffusion of online dating. Following an introduction on
its principles and functioning, we apply a rational choice approach
to assess the consequences of online dating on mate choice processes
from a theoretical standpoint. To give an illustration of the concept
of competition for attention, and of the logics of exchange, network
externalities, supply and demand, we take advantage of the \textquotedblleft marketplace\textquotedblright{}
metaphor. An empirical analysis on Pew Research Center's \textquotedblleft Gaming,
Jobs and Broadband\textquotedblright{} 2015 data set is run to test
two hypotheses on the usage of online dating. Results are then discussed
in the light of prior extensive research. We find that those who are
more likely to be online daters are mostly young White males, with
above-average socioeconomic status, a university degree and more liberal
views. Thus, based solely on the available data, we dismiss the hypothesis
of the primacy of online dating for those in a thin market for potential
partners.
