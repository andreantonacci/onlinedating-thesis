
\lhead[\leftmark]{\leftmark}

\rhead[\leftmark]{}

\lfoot{}

\cfoot{}

\cfoot[\thepage]{\thepage}

\chapter{Methods and Data}

\section{Research Design and Data Set}

In accordance with the above-mentioned insights, we elaborate some
hypotheses on online dating users and usage to test in this study.
As described in section \ref{sec:A-Solution-to}, the increased search
efficiency is especially convenient to those people who lack an easy
access to a partner market \citep[e.g.,][]{Stevenson2007Marriage-and-Di},
thus a first hypothesis can be outlined.
\begin{labeling}{000.000.0000000}
\item [{\textsl{Hypothesis\ 1}:}] Online daters are more likely to be
in a thin market for partners or in a minority group (e.g., race,
sexual orientation, religion, disability, etc.).
\end{labeling}
On the other hand, we investigate the online dating platforms penetration
among all social strata and the upper ones in particular. In fact,
such demographic groups (young adults, graduates and those from high-income
households) are those who use the Internet the most\footnote{As reported in the Internet/Broadband Fact Sheet by Pew Research Center
(2018).}. Thus, consistent with prior research \citep[e.g.,][]{Cacioppo2013Marital-satisfa},
we expect the following traits to be confirmed:
\begin{labeling}{000.000.0000000}
\item [{\textsl{Hypothesis\ 2}:}] Young, white males, who are not already
married nor with children, with a higher socioeconomic status and
liberal views are those who use online dating more often.
\end{labeling}
Because official data from online dating private companies is not
made publicly available, all analyses in this thesis are based on
secondary data from Pew Research Center\textquoteright s Internet,
Science \& Technology Project \textquotedblleft Spring Tracking\textquotedblright{}
Survey of 2015. Among the many topics covered \textendash{} such as
gaming, home broadband and smartphone usage \textendash{} online dating
questions were also included.

The survey was conducted between June and July 2015 and provides a
national sample of 2,001 adults (18 years of age or older) living
in the United States. 701 of these respondents have been interviewed
on a landline telephone, while the remaining 1,300 were interviewed
on a cellphone, all via random digit dialing. Therefore, respondents
in the landline sample have been randomly picked by asking for the
youngest adult who was at home at the moment of calling. Conversely,
those in the cellphone sample were personally interviewed when they
answered the phone, if they were at least 18 years old. With this
method, fraudulent or deceptive responding is minimised: for instance,
no multiple surveys could be completed from the same respondent, nor
they could complete it too quickly to reflect valid data (i.e., random
responses).

Data collection processes were conducted with strict methodology,
and the final data set has been provided with full documentation.
As a result of the rigorous techniques put in place, the data set
is considered to be fully representative of all residents in each
U.S. state. In order to correct for the stratification sampling design
at the state level and to overcome over-sampling problems due to different
population densities and other known biases (on parameters such as
age, race, gender, etc.), we conduct all analyses weighting data accordingly.
Moreover, the weights take into account that people in large households
and that people with both a landline and a cellphone have respectively
lower and higher chances of being selected. All analyses and computations
are processed with IBM SPSS software. 

As we wish to test the aforementioned hypotheses by measuring the
partial (ceteris paribus) effects of each independent variable on
the variation in the usage of online dating, we run a Multiple Regression
Model (MRM) along with some preliminary descriptive statistics analyses.
We develop different models to explain the online dating usage frequency,
progressively adding up variables to reach our final and complete
model (see \prettyref{tab:MRM-benchmark}). All analyses are conducted
on the same observations and with the same dependent variable, that
is thoroughly discussed in the following section.

\section{Online Dating Usage: Dependent Variable}

We want to investigate the impact of different independent variables
on the actual usage of online dating platforms. Respondents were asked
two questions that particularly fit our needs: \textsc{date1a} (\textquotedblleft Have
YOU, personally, ever used an online dating site such as Match.com,
eHarmony, or OK Cupid?\textquotedblright ), that has been asked to
all Internet users (\textsc{eminuse}\footnote{\textsc{eminuse} (asked all): \textquotedblleft Do you use the internet
or email, at least occasionally?\textquotedblright{} }=1 or \textsc{intmob}\footnote{\textsc{intmob} (asked all): \textquotedblleft Do you access the internet
on a cell phone, tablet or other mobile handheld device, at least
occasionally?\textquotedblright{}}=1), and \textsc{date2a} (\textquotedblleft Have you ever used a dating
app on your cell phone?\textquotedblright ), that has been asked to
every smartphone owner (\textsc{smart1}\footnote{\textsc{smart1} (asked if have a cell phone): \textquotedblleft Some
cell phones are \textquoteleft smartphones\textquoteright{} because
of certain features they have. Is your cell phone a smartphone such
as an iPhone, Android, Blackberry or Windows phone, or are you not
sure?\textquotedblright{}}=1). Both variables can take either value 1 (\textquotedblleft Yes\textquotedblright ),
2 (\textquotedblleft No\textquotedblright ), 8 (\textquotedblleft Don't
know\textquotedblright ) or 9 (\textquotedblleft Refused\textquotedblright ).
As explained later, they have been jointly used to create a new dummy
(\textsc{isonlinedater}) that is included in the MRM as the dependent
variable. 

As we would expect, they both showed high percentages of missing values:
13.4\% for \textsc{date1a} and 32.4\% for\textsc{ date2a} (see Table
\ref{tab:Univariate-Statistics-dependent} and \ref{tab:Missing-Value-Analysis};
Figures \ref{fig:date1a:-frequency-bar} and \ref{fig:date2a:-frequency-bar}).
However, in the light of the fact that there is no commonly accepted
way to handle missing data, and that a generally high percentage of
them can be found in social science research papers, we decided to
proceed by keeping all cases in further analyses. To clarify, we could
have adopted other solutions: for instance, we could have omitted
all cases with missing values for both \textsc{date1a} and \textsc{date2a}
(listwise deletion) \textendash{} and thus keeping those cases that
reported a 1 (\textquotedblleft Yes\textquotedblright ) in at least
one of these variables\footnote{I.e., meaning that the respondent had used an online dating site at
least once but never a smartphone dating app (or vice versa), hence
he or she should be considered as an \textquotedblleft online dater\textquotedblright{}
anyway.} \textendash{} which should be considered as \textquotedblleft online
daters\textquotedblright{} anyway; filtering out all respondents who
reported not to use the Internet at least occasionally (\textsc{eminuse}$\neq$1)
could have also been a quick and similar fix. However, none of these
proved to be sound choices. In fact, we want to take into account
in our analysis also people who do not frequently use the Internet
(namely the elderly). We suspect that the great majority of missing
values is not due to an error, but rather to the fact that these questions
have been asked only to a restricted pool of respondents: those who
had disclosed that they use the Internet or own a smartphone. In fact,
the Little's MCAR Test is significant at 1\%, thus we reject the null
hypothesis that values are missing completely at random (see Table
\ref{tab:Missing-Value-Analysis}). That said, we do not exclude the
occurrence of any other potential issue a priori, in fact some improbable
values have also been observed, but their frequency seems negligible
(see Table \ref{tab:Crosstabs:-date1a-and}). Since no other issues
are noticeable, we proceed by describing the actual dependent variable
of our model.

In order to run a regression on the usage of online dating, we computed
a new dummy variable \textsc{isonlinedater} that takes value 1 if
either \textsc{date1a} or \textsc{date2a} is true, 0 otherwise. In
this manner, we are considering all missing values from \textsc{date1a}
and \textsc{date2a} as if they were coded a 2 (\textquotedblleft No\textquotedblright )\footnote{This provides a conservative estimate since we are not taking into
account potential respondents who lied by not declaring their actual
usage of online dating.}. Thus, from here on, we define as \textquotedblleft online dater\textquotedblright{}
anyone who has used a dating site or a mobile dating app at least
once. As a result, 15.5\%\footnote{Notice that this slightly higher percentage of 1 (\textquotedblleft Yes\textquotedblright )
in \textsc{isonlinedater}, compared to \textsc{date1a} and \textsc{date2a},
is due to the fact that some respondents who reported a 2 (\textquotedblleft No\textquotedblright )
in one of these variables are being considered anyway \textquotedblleft online
daters\textquotedblright , if they reported a 1 (\textquotedblleft Yes\textquotedblright )
in the other one. } of respondents is defined as \textquotedblleft online dater\textquotedblright{}
(see Table \ref{tab:isonlinedater:-frequency-table.} and Figure \ref{fig:isonlinedater:-frequency-bar}).
It follows that \textsc{isonlinedater} has been employed as the outcome
variable in our final regression model. Table \ref{tab:Univariate-Statistics-dependent}
collects some statistics for the variables in question.

\begin{table}[H]
\renewcommand{\arraystretch}{1.4}
\begin{centering}
\begin{tabular}{lllccc}
\hline 
 &  &  & \textsc{date1a} & \textsc{date2a} & \textsc{isonlinedater}\tabularnewline
\hline 
\hline 
\textbf{N} & \textbf{Valid} &  & 5428  & 4236 & 6267\tabularnewline
 & \textbf{Missing} &  & 839 & 2032 & 0\tabularnewline
\textbf{Mean} &  &  & 1.87 & 1.89 & .1549\tabularnewline
\textbf{Median} &  &  & 2.00 & 2.00 & .0000\tabularnewline
\textbf{Std. Deviation} &  &  & .442 & .445 & .36181\tabularnewline
\textbf{Skewness} &  &  & 4.643 & 4.982 & 1.908\tabularnewline
\multicolumn{2}{l}{\textbf{Std. Error of Skewness}} &  & .033 & .038 & .031\tabularnewline
\textbf{Kurtosis} &  &  & 88.310 & 81.147 & 1.642\tabularnewline
\multicolumn{2}{l}{\textbf{Std. Error of Kurtosis}} &  & .066 & .075 & .062\tabularnewline
\textbf{Minimum} &  &  & 1 & 1 & .00\tabularnewline
\textbf{Maximum} &  &  & 9 & 8 & 1.00\tabularnewline
\hline 
\end{tabular}\bigskip{}
\par\end{centering}
\begin{centering}
\caption{Univariate Statistics for the dependent variables.\label{tab:Univariate-Statistics-dependent} }
\par\end{centering}
\end{table}

\section{Independent Variables}

Unless otherwise noted, all the covariates have been recoded to provide
better interpretable results. The following input variables have been
standardised for consistency reasons, and only pertinent answers have
been included in our analyses. In other words, common entries such
as 98 (\textquotedblleft Don\textquoteright t know\textquotedblright )
or 99 (\textquotedblleft No answer\textquotedblright ) have been recoded
as system-missing values.

As regards the controlling variables, we included some basic regressors
in accordance to preexisting literature. \textsc{gender} is a dummy
variable that takes value 0 if male, 1 if female. \textsc{university}
and \textsc{minrace} also follow the same logic: they take value 1
in case the respondent \textendash{} respectively \textendash{} graduated
at least at a Bachelor level and is from a racial minority (i.e.,
is not White), 0 otherwise. Since most of the covariates are categorical,
nominal or ordinal variables (\textsc{age} is the only numerical variable
included in our model), new dummies have been coded for them, following
the general principle of assigning a 1 for the presence of a level,
and a 0 for its absence. Some anchored scales are also present. \textsc{internetfreq}\footnote{\textsc{internetfreq} is the inverted anchor scale of the original
variable \textsc{intfreq}, that has been recoded for more easily interpretable
results.} (\textquotedblleft About how often do you use the internet?\textquotedblright )
and \textsc{polview} (\textquotedblleft In general, would you describe
your political views as...\textquotedblright ) are both 5-point scales
that range from, respectively, \textquotedblleft Less often\textquotedblright{}
to \textquotedblleft Almost constantly\textquotedblright{} and from
\textquotedblleft Very conservative\textquotedblright{} to \textquotedblleft Very
liberal\textquotedblright . Because we did not want to lose the information
in their ordering, we opted for treating them as numeric. However,
this was not possible for the original variable \textsc{inc} (\textquotedblleft Last
year - that is in 2014 - what was your total family income from all
sources, before taxes?\textquotedblright ), which is also an anchored
scale, but it ranges from 1 (\textquotedblleft Less than \$10,000\textquotedblright )
to 9 (\textquotedblleft \$150,000 or more\textquotedblright ) with
different numerical distance between each category. Thus, we introduced
new dummies, \textsc{income1}, \textsc{income2} and \textsc{income3},
that split respondents into three different income categories, from
the lowest to the highest. 

Table \ref{tab:Description-of-variables-selected} provides a full
list of selected variables included in the final regression model.
Since all these variables directly refer to respondent\textquoteright s
individual traits and preferences, and interpretation seems straightforward,
we do not deepen further into them.\clearpage{}\pagestyle{plain}
\begin{sidewaystable}
\centering{}%
\begin{minipage}[t]{1\columnwidth}%
\begin{center}
\renewcommand{\arraystretch}{1.2}%
\begin{tabular}{lll}
\hline 
\textbf{Label} & \textbf{Description or original question} & \textbf{Values admitted (recoded)}\tabularnewline
\hline 
\hline 
\textsc{\small{}isonlinedater} & {\small{}Have you used a dating site or a mobile dating app at least
once?} & {\small{}0=No; 1=Yes.}\tabularnewline
\textsc{\small{}gender} & {\small{}Respondent's sex.} & {\small{}0=Male; 1=Female.}\tabularnewline
\textsc{\small{}age} & {\small{}What is your age? } & \textsl{\small{}numeric}\tabularnewline
\textsc{\small{}university} & {\small{}Is true if respondent graduated at least at a Bachelor level.} & {\small{}0=No university degree; 1=Yes.}\tabularnewline
\textsc{\small{}minrace} & {\small{}Is true if respondent is from a racial minority (not White).} & {\small{}0=White; 1=Else.}\tabularnewline
\multirow{1}{*}{\textsc{\small{}income1}} & {\small{}Was your family income last year from less than \$10,000
to under \$30,000?} & \multirow{1}{*}{{\small{}0=No; 1=Yes.}}\tabularnewline
\multirow{1}{*}{\textsc{\small{}income2}} & {\small{}Was your family income last year from \$30,000 to under \$75,000?} & \multirow{1}{*}{{\small{}0=No; 1=Yes.}}\tabularnewline
\textsc{\small{}income3} & {\small{}Was your family income last year \$75,000 or more?} & {\small{}0=No; 1=Yes.}\tabularnewline
\multirow{2}{*}{\textsc{\small{}married}} & \multirow{2}{*}{{\small{}Are you currently married or living with a partner?}} & {\small{}0=Never been married;}\tabularnewline
 &  & {\small{}1=Married/living with a partner.}\tabularnewline
\textsc{\small{}isparent} & {\small{}Are you parent of any children under 18?} & {\small{}0=No; 1=Yes.}\tabularnewline
\multirow{2}{*}{\textsc{\small{}polview}} & \multirow{2}{*}{{\small{}Would you describe your political view as...}} & {\small{}1=Very conservative; 2=Conservative;}\tabularnewline
 &  & {\small{}3=Moderate; 4=Liberal; 5=Very liberal.}\tabularnewline
\textsc{\small{}urban} & {\small{}Community type is urban or suburban.} & {\small{}0=Rural; 1=Urban/Suburban}\tabularnewline
\textsc{\small{}physicallabour} & {\small{}Respondent's job involves manual labour.} & {\small{}0=No; 1=Yes.}\tabularnewline
\textsc{\small{}disability} & {\small{}Respondent has any handicap or disability.} & {\small{}0=No; 1=Yes.}\tabularnewline
\multirow{3}{*}{\textsc{\small{}internetfreq}} & \multirow{3}{*}{{\small{}How often do you use the Internet?}} & {\small{}1=Less often; 2=Several times a week;}\tabularnewline
 &  & {\small{}3=About once a day; 4= Several times}\tabularnewline
 &  & {\small{}a day; 5=Almost constantly.}\tabularnewline
\textsc{\small{}attitude1} & {\small{}Online dating is a good way to meet people.} & {\small{}0=Disagree; 1=Agree.}\tabularnewline
\textsc{\small{}attitude2} & {\small{}Online dating is easier and efficient.} & {\small{}0=Disagree; 1=Agree.}\tabularnewline
\hline 
\end{tabular}
\par\end{center}
\begin{center}
\caption{Description of variables selected in the final model.\label{tab:Description-of-variables-selected}}
\par\end{center}%
\end{minipage}
\end{sidewaystable}
\clearpage{}\pagestyle{fancy}

\section{Model Construction}

Among all the variables provided in the data set, we decided to include
only those ones that best could influence individual adoption or usage
of online dating in our opinion. The first model is a very basic one,
with only some controlling variables (socio-demographic) included:
\textsc{gender}, \textsc{age}, \textsc{university}, \textsc{minrace},
\textsc{income2} and \textsc{income3}. To avoid the dummy variable
trap, we kept \textsc{income1} out from our models as a reference
category. Our second model, takes into account also some attributes
of the life course, like marital history or dependent children (underage).
The third model builds upon the first ones and considers political
ideologies and potential handicaps as well. It includes \textsc{polview}
and \textsc{disability}. The final\footnote{Actually, a further model with \textsc{physicallabour} is also presented.
The rationale for including it is to assess to what extent having
a manual or physical intensive job position affects the outcome variable
in question. However, as it is explained later, it lacks of significance
despite a higher $R^{2}$, thus we discourage considering this model
as the definitive one. } and most comprehensive model includes these previous variables, along
with other key ones to account for individual attitudes and preferences
toward the Internet and technology in general. Thus it turned out
to be based upon the following regressors:
\begin{multline}
Y=\beta_{0}+\beta_{1}\cdot\mathtt{\mathsf{\mathit{GENDER}}}+\beta_{2}\cdot AGE+\beta_{3}\cdot UNIVERSITY+\beta_{4}\cdot MINRACE\\
+\beta_{5}\cdot INCOME2+\beta_{6}\cdot INCOME3+\beta_{7}\cdot MARRIED+\beta_{8}\cdot ISPARENT\\
+\beta_{9}\cdot POLVIEW+\beta_{10}\cdot DISABILITY+\beta_{11}\cdot INTERNETFREQ\\
+\beta_{12}\cdot ATTITUDE1+\beta_{13}\cdot ATTITUDE2+\varepsilon_{i}\qquad\hfill\label{eq:final-model}
\end{multline}

One simpler model has also been developed, specifically to test the
thin market \textsl{hypothesis 1}. Notwithstanding that very few
useful variables can be found in the data set for this purpose \textendash{}
questions on religious beliefs and sexual orientation, for instance,
could have definitely improved this model \textendash{} we included
\textsc{disability} and \textsc{minrace}. We also decided to add \textsc{urban,
}in order to assess whether living in a rural community might push
inhabitants to resort to online dating for their mating purposes,
since the low population density could be equated with being in a
thin market for potential partners.


