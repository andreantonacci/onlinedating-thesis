
\lhead[\leftmark]{\leftmark}

\rhead[\leftmark]{}

\lfoot{}

\rfoot{}

\cfoot[\thepage]{\thepage}

\chapter{Conclusion}

This thesis offers insight into the principles of online dating and
explores the determinants of its usage through simple models. We investigated
two \textendash{} rather opposing \textendash{} views and tested their
hypotheses. While we found little empirical evidence supporting the
hypothesis of online dating primacy for those in a thin market for
romantic mates, we recognise that some common distinguishing features
can be traced among its users. Through our final model we were able
to identify the key demographics, traits and attitudes of those who
reported having utilised online dating at least once. In the light
of the resulting quite weak (yet significant) models, we would suggest
grounding future research on more recent data, as well as taking into
account personality traits and further distinctions among cases (e.g.,
sexual orientation, engagement in religious practices, intentions
and desired outcomes from the dating platforms above all).

Ultimately, online dating exists to allow a more efficient search
for a partner, but it is paramount to acknowledge that \textquotedblleft the
Internet complements, rather than displaces, existing behaviour patterns\textquotedblright{}
\citep{DiMaggio2001Social-Implicat}, and by extension, so do the
tools it made available, like online dating platforms. In conclusion,
we believe that more accurate research should be carried out, notwithstanding
the fact that it will take generations, perhaps, to assess the impact
of such practices at societal level \citep{Stevenson2007Marriage-and-Di}.
The Internet might be able to change the marriage market by promoting
better matches, but it still is too early to appraise its effects
on divorce and marriage rates, for instance, and ultimately on society
of the generations to come.


